${\bf f_{1}}$ & \multicolumn{2}{@{}c@{}}{4.0 \quad} & \multicolumn{2}{@{}c@{}}{8.0 \quad} & \multicolumn{2}{@{}c@{}}{8.0 \quad} & \multicolumn{2}{@{}c@{}}{8.0 \quad} & \multicolumn{2}{@{}c@{}}{8.0 \quad} & \multicolumn{2}{@{}c@{}}{8.0 \quad} & \multicolumn{2}{@{}c|@{}}{8.0} & 15 & /15\\
 & 1 & .5(0.6) & 1& & 1& & 1& & 1& & 1& & 1& & 15 & /15\\