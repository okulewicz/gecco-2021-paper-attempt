\documentclass[sigconf]{acmart}

\usepackage{booktabs} % For formal tables
\usepackage{graphicx}
\usepackage{rotating}
\usepackage{tabularx}
\usepackage{xstring} % for string operations
\usepackage{wasysym} % Table legend with symbols input from post-processing
\usepackage{MnSymbol} % Table legend with symbols input from post-processing
\usepackage{float}
\usepackage{ifthen}

\usepackage{algorithm}% <=================http://ctan.org/pkg/algorithms
\usepackage{algpseudocode}% <========== http://ctan.org/pkg/algorithmicx

% define some COCO/dvipsnames colors because
% ACM style does not allow to use them directly
%\definecolor{NavyBlue}{HTML}{000080}
%\definecolor{Magenta}{HTML}{FF00FF}
%\definecolor{Orange}{HTML}{FFA500}
%\definecolor{CornflowerBlue}{HTML}{6495ED}
%\definecolor{YellowGreen}{HTML}{9ACD32}
%\definecolor{Gray}{HTML}{BEBEBE}
%\definecolor{Yellow}{HTML}{FFFF00}
%\definecolor{GreenYellow}{HTML}{ADFF2F}
%\definecolor{ForestGreen}{HTML}{228B22}
%\definecolor{Lavender}{HTML}{FFC0CB}
%\definecolor{SkyBlue}{HTML}{87CEEB}
%\definecolor{NavyBlue}{HTML}{000080}
%\definecolor{Goldenrod}{HTML}{DDF700}
%\definecolor{VioletRed}{HTML}{D02090}
%\definecolor{CornflowerBlue}{HTML}{6495ED}
%\definecolor{LimeGreen}{HTML}{32CD32}


% Copyright
%\setcopyright{none}
%\setcopyright{acmcopyright}
%\setcopyright{acmlicensed}
\setcopyright{rightsretained}
%\setcopyright{usgov}
%\setcopyright{usgovmixed}
%\setcopyright{cagov}
%\setcopyright{cagovmixed}


% DOI
\acmDOI{10.1145/123_4}

% ISBN
\acmISBN{123-4567-24-567/18/07}

%Conference
\acmConference[GECCO '21]{the Genetic and Evolutionary Computation Conference 2021}{July 10--14, 2019}{Lille, France}
\acmYear{2021}
\copyrightyear{2021}

\acmPrice{15.00}



%%%%%%%%%%%%%%%%%%%%%%   END OF PREAMBLE   %%%%%%%%%%%%%%%%%%%%%%%%%%%%%%%%%%%%

%%%%%%%%%%%%%%%%%%%%%%%%%%%%%%%%%%%%%%%%%%%%%%%%%%%%%%%%%%%%%%%%%%%%%%%%%%%%%%%
%%%%%%%%% TO BE EDITED %%%%%%%%%%%%%%%%%%%%%%%%%%%%%%%%%%%%%%%%%%%%%%%%%%%%%%%%
%%%%%%%%%%%%%%%%%%%%%%%%%%%%%%%%%%%%%%%%%%%%%%%%%%%%%%%%%%%%%%%%%%%%%%%%%%%%%%%
% specify acronyms for algorithm1 (1st arg. of post-processing) and algorithm2 (2nd arg.) 
%\newcommand{\algorithmA}{algorithmB}  % first argument in the post-processing
%\newcommand{\algorithmB}{algorithmB}  % second argument in the post-processing
% for the short acronyms in the tables, adjust the following to lines if required.
%\newcommand{\algorithmAshort}{algA}  % first argument in the post-processing
%\newcommand{\algorithmBshort}{algB}  % second argument in the post-processing

% rungeneric.py writes data into a subfolder of ppdata
\newcommand{\bbobdatapath}{ppdata3/} % change default output folder of COCO if desired
\input{\bbobdatapath cocopp_commands.tex} % provide default of algname and algfolder
%%%%%%%%%%%%%%%%%%%%%%%%%%%%%%%%%%%%%%%%%%%%%%%%%%%%%%%%%%%%%%%%%%%%%%%%%%%%%%%
%%%%%%%%%%%%%%%%%%%%%%%%%%%%%%%%%%%%%%%%%%%%%%%%%%%%%%%%%%%%%%%%%%%%%%%%%%%%%%%
%%%%%%%%%%%%%%%%%%%%%%%%%%%%%%%%%%%%%%%%%%%%%%%%%%%%%%%%%%%%%%%%%%%%%%%%%%%%%%%
\graphicspath{{\bbobdatapath\algsfolder}}

% pre-defined commands
\newcommand{\DIM}{\ensuremath{\mathrm{DIM}}}
\newcommand{\ERT}{\ensuremath{\mathrm{ERT}}}
\newcommand{\FEvals}{\ensuremath{\mathrm{FEvals}}}
\newcommand{\nruns}{\ensuremath{\mathrm{Nruns}}}
\newcommand{\Dfb}{\ensuremath{\Delta f_{\mathrm{best}}}}
\newcommand{\Df}{\ensuremath{\Delta f}}
\newcommand{\nbFEs}{\ensuremath{\mathrm{\#FEs}}}
\newcommand{\fopt}{\ensuremath{f_\mathrm{opt}}}
\newcommand{\ftarget}{\ensuremath{f_\mathrm{t}}}
\newcommand{\CrE}{\ensuremath{\mathrm{CrE}}}
\newcommand{\change}[1]{{\color{red} #1}}


\begin{document}

\title{Benchmarking SHADE algorithm enhanced with model based optimization on the BBOB noiseless testbed}
\renewcommand{\shorttitle}{SHADE algorithm enhanced with model based optimization on BBOB}

\author{Michał Okulewicz}
%\authornote{tba if needed}
%\orcid{1234-5678-9012}
\affiliation{%
  \institution{Faculty of Mathematics and Information Science\\
  Warsaw University of Technology, Poland}
%  \streetaddress{P.O. Box 1212}
%  \city{Dublin} 
%  \state{Ohio} 
%  \postcode{43017-6221}
}
\email{M.Okulewicz@mini.pw.edu.pl}
%
\author{Mateusz Zaborski}
%\authornote{The secretary disavows any knowledge of this author's actions.}
\affiliation{%
  \institution{Faculty of Mathematics and Information Science\\
  Warsaw University of Technology, Poland}
%  \streetaddress{P.O. Box 1212}
%  \city{Dublin} 
%  \state{Ohio} 
%  \postcode{43017-6221}
}
\email{M.Zaborski@mini.pw.edu.pl}
%
%\author{Lars Th{\o}rv{\"a}ld}
%\authornote{This author is the
%  one who did all the really hard work.}
%\affiliation{%
%  \institution{The Th{\o}rv{\"a}ld Group}
%  \streetaddress{1 Th{\o}rv{\"a}ld Circle}
%  \city{Hekla} 
%  \country{Iceland}}
%\email{larst@affiliation.org}
%
%\author{Lawrence P. Leipuner}
%\affiliation{
%  \institution{Brookhaven Laboratories}
%  \streetaddress{P.O. Box 5000}}
%\email{lleipuner@researchlabs.org}
%
%\author{Sean Fogarty}
%\affiliation{%
%  \institution{NASA Ames Research Center}
%  \city{Moffett Field}
%  \state{California} 
%  \postcode{94035}}
%\email{fogartys@amesres.org}
%
%\author{Charles Palmer}
%\affiliation{%
%  \institution{Palmer Research Laboratories}
%  \streetaddress{8600 Datapoint Drive}
%  \city{San Antonio}
%  \state{Texas} 
%  \postcode{78229}}
%\email{cpalmer@prl.com}
%
%\author{John Smith}
%\affiliation{\institution{The Th{\o}rv{\"a}ld Group}}
%\email{jsmith@affiliation.org}
%
%\author{Julius P.~Kumquat}
%\affiliation{\institution{The Kumquat Consortium}}
%\email{jpkumquat@consortium.net}

% The default list of authors is too long for headers}
\renewcommand{\shortauthors}{Firstname Lastname et. al.}


\begin{abstract}
 In this paper we evaluate the SHADE-LM algorithm on the BBOB noiseless testbed.
 The algorithm hybridizes the SHADE algorithm with a model based optimization.
 This hybridization is performed in a transparent manner for both optimizers,
 with SHADE having access to the samples improved by model based optimization,
 and models of square function are fitted on the current population.
 The paper compares this extended version with the performance of original
 R-SHADE and KL-BIPOP-CMAES.
\end{abstract}


%
% The code below should be generated by the tool at
% http://dl.acm.org/ccs.cfm
% Please copy and paste the code instead of the example below. 
%
 \begin{CCSXML}
<ccs2012>
<concept>
<concept_id>10010147.10010178.10010205.10010208</concept_id>
<concept_desc>Computing methodologies~Continuous space search</concept_desc>
<concept_significance>500</concept_significance>
</concept>
</ccs2012>
\end{CCSXML}

\ccsdesc[500]{Computing methodologies~Continuous space search}


% We no longer use \terms command
%\terms{Algorithms}

% Complete with anything that is needed
\keywords{Benchmarking, Black-box optimization}

\maketitle


\section{Introduction}

R-SHADE \cite{Tanabe2014} algorithm has been proposed as one of the more successful
modification of Differential Evolution, following the path of adapting the scale
and cross-over probability factors, employing the archive of previous best samples,
utilizing the current-to-best position update and restart mechanism based on population
locations or values spread.

Hybridizing plain Particle Swarm Optimization (PSO) and Differential Evolution (DE)
algorithms with model based optimizer has already been proved
to improve their performance \cite{zaborski2019generalized,zaborski2020analysis,Okulewicz2020}.

In this paper we utilize our Generalized Adaptive Particle Swarm Optimization (GAPSO) framework
\cite{ulinski2018generalized,Okulewicz2020}
and implement within it a version of SHADE \cite{Tanabe2014} and model based optimizer \cite{zaborski2020analysis}.
%
\section{GAPSO Framework Concept}
The concept of GAPSO framework is to allow for hybridization of optimization algorithms,
in a way which will be transparent to the hybridized methods.
This approach comes from the observation that methods such as PSO or DE,
need only a minimal amount of information (i.e. locations and values
of the previously sampled locations). Please note that PSO's velocity
is simply a difference between previous and current location.
Therefore an algorithm which stores current, previous and best location
for each individual (particle) allows to employ sampling of both DE and PSO
based algorithms, in a transparent manner from the point of view of individual.

\section{SHADE-LM Algorithm Presentation}

\begin{algorithm}[H]
	\begin{algorithmic}[1]
	\footnotesize
  \State $F$ is optimized $\mathbb{R}^n \rightarrow \mathbb{R}$ function, $Bounds$ is an $\left(\mathbb{R}^2\right)^n$ vector
  \State $Swarm$ is a set of PSO particles, $Behavior$ is particle's velocity update rule
  \State $Initializer$ is particle's initial location sampler
  \State $SamplesArchive$ is an RTree based samples' index
  \State $BehaviorAdapter$ collects optimum value improvement data
  \State $RestartManager$ observes swarm state and performance
  \State $Bounds \gets f.getBounds()$ \Comment {Initially the whole area is considered}
  \State $PerformanceMonitor.behaviourProbabilities \gets initialProbabilities$
  \State $LocalOptima \gets \emptyset$ \Comment {Set of optima estimations}
  \While {Stopping criterion not met}
	  \For {$Particle \in Swarm$}
		  \State $Particle.x \gets Initializer.nextSample(Bounds)$
		  \State $Sample \gets F.evaluate(Particle.x)$
		  \State $BehaviorAdapter.registerValue(Sample)$
	  \EndFor
	  \For {$Particle \in Swarm$}
		  \State $Particle.v \gets \dfrac{(Particle_i.x - Particle_j.x)}{2.0}$
	  \EndFor
	  \While {$RestartManager.shouldOptimizationContinue(Swarm)$}
		  \For {$Particle \in Swarm$}
			  \State $Behaviour \gets BehaviorAdapter.sampleBehaviourPool()$ \Comment {Mixing behaviors \label{line:alg:GAPSO:mixing}}
			  \State $Particle.v \gets Behavior.computeVelocity(Particle,Swarm,Archive)$
			  \State $Particle.x \gets Particle.x + Particle.v$
			  \If {$SamplesArchive.stored(Particle.x)$}
				  \State $Sample \gets SamplesArchive.retrieve(Particle.x)$
			  \Else
				  \State $Sample \gets F.evaluate(Particle.x)$
				  \State $SamplesArchive.store(Sample)$
			  \EndIf
			  \State $BehaviorAdapter.registerImprovement(Sample,Behaviour)$
		  \EndFor
		  \State $BehaviorAdapter.recomputeBehaviourProbabilities()$
	  \EndWhile
	  \State $LocalOptima \gets LocalOptima \cup Swarm.bestSample$
	  \State $Bounds \gets Initializer(LocalOptima)$ \Comment {Guiding search process}
  \EndWhile
  \caption{M-GAPSO high--level pseudocode%
  \label{alg:GAPSO}}
  \end{algorithmic}
  \end{algorithm}


%
\section{Experimental Procedure}
%
%%%%%%%%%%%%%%%%%%%%%%%%%%%%%%%%%%%%%%%%%%%%%%%%%%%%%%%%%%%%%%%%%%%%%%%%%%%%%%%
\section{CPU Timing}
%%%%%%%%%%%%%%%%%%%%%%%%%%%%%%%%%%%%%%%%%%%%%%%%%%%%%%%%%%%%%%%%%%%%%%%%%%%%%%%
% note that the following text is just a proposal and can/should be changed to your needs:
In order to evaluate the CPU timing of the algorithm, we have run the {SHADE-LM}
on the {bbob test suite \cite{hansen2010fun}} with restarts for a maximum budget
equal to {$10^6D)$} function evaluations according to \cite{hansen2016exp}.
The {Java} code was run on  single core of a {Windows Intel(R) Core(TM) i7-9750H CPU @ 2.60GHz}.
The time per function evaluation for dimensions 2, 3, 5, 10, 20, 40 equals \change{$x.x$}, \change{$x.x$}, \change{$x.x$}, \change{$xx$}, \change{$xxx$}\change{, and $xxx$} seconds respectively. 

%%%%%%%%%%%%%%%%%%%%%%%%%%%%%%%%%%%%%%%%%%%%%%%%%%%%%%%%%%%%%%%%%%%%%%%%%%%%%%%
\section{Results}
%%%%%%%%%%%%%%%%%%%%%%%%%%%%%%%%%%%%%%%%%%%%%%%%%%%%%%%%%%%%%%%%%%%%%%%%%%%%%%%

Results from experiments according to \cite{hansen2016exp} and \cite{hansen2016perfass} on the
benchmark functions given in \cite{wp200901_2010,hansen2010fun} are
presented in Figures~\ref{fig:scaling}, \ref{fig:ECDFs05D} and
\ref{fig:ECDFs20D} and in Tables~\ref{tab:ERTs5} and~\ref{tab:ERTs20}.
The experiments were performed with COCO \cite{hansen2020cocoplat}, version
{2.3}, the plots were produced with version {2.4}.

The \textbf{expected runtime (ERT)}, used in the figures and tables,
depends on a given target function value, $\ftarget=\fopt+\Df$, and is
computed over all relevant trials as the number of function
evaluations executed during each trial while the best function value
did not reach \ftarget, summed over all trials and divided by the
number of trials that actually reached \ftarget\
\cite{hansen2012exp,price1997dev}.  \textbf{Statistical significance}
is tested with the rank-sum test for a given target $\Delta\ftarget$
%($10^{-8}$ as in Figure~\ref{fig:scaling})
using, for each trial,
either the number of needed function evaluations to reach
$\Delta\ftarget$ (inverted and multiplied by $-1$), or, if the target
was not reached, the best $\Df$-value achieved, measured only up to
the smallest number of overall function evaluations for any
unsuccessful trial under consideration.


%%%%%%%%%%%%%%%%%%%%%%%%%%%%%%%%%%%%%%%%%%%%%%%%%%%%%%%%%%%%%%%%%%%%%%%%%%%%%%%
%%%%%%%%%%%%%%%%%%%%%%%%%%%%%%%%%%%%%%%%%%%%%%%%%%%%%%%%%%%%%%%%%%%%%%%%%%%%%%%

% Scaling of ERT with dimension

%%%%%%%%%%%%%%%%%%%%%%%%%%%%%%%%%%%%%%%%%%%%%%%%%%%%%%%%%%%%%%%%%%%%%%%%%%%%%%%
\begin{figure*}
\centering
\begin{tabular}{@{}c@{}c@{}c@{}c@{}}
\includegraphics[width=0.238\textwidth]{ppfigs_f001}&
\includegraphics[width=0.238\textwidth]{ppfigs_f002}&
\includegraphics[width=0.238\textwidth]{ppfigs_f003}&
\includegraphics[width=0.238\textwidth]{ppfigs_f004}\\
\includegraphics[width=0.238\textwidth]{ppfigs_f005}&
\includegraphics[width=0.238\textwidth]{ppfigs_f006}&
\includegraphics[width=0.238\textwidth]{ppfigs_f007}&
\includegraphics[width=0.238\textwidth]{ppfigs_f008}\\
\includegraphics[width=0.238\textwidth]{ppfigs_f009}&
\includegraphics[width=0.238\textwidth]{ppfigs_f010}&
\includegraphics[width=0.238\textwidth]{ppfigs_f011}&
\includegraphics[width=0.238\textwidth]{ppfigs_f012}\\
\includegraphics[width=0.238\textwidth]{ppfigs_f013}&
\includegraphics[width=0.238\textwidth]{ppfigs_f014}&
\includegraphics[width=0.238\textwidth]{ppfigs_f015}&
\includegraphics[width=0.238\textwidth]{ppfigs_f016}\\
\includegraphics[width=0.238\textwidth]{ppfigs_f017}&
\includegraphics[width=0.238\textwidth]{ppfigs_f018}&
\includegraphics[width=0.238\textwidth]{ppfigs_f019}&
\includegraphics[width=0.238\textwidth]{ppfigs_f020}\\
\includegraphics[width=0.238\textwidth]{ppfigs_f021}&
\includegraphics[width=0.238\textwidth]{ppfigs_f022}&
\includegraphics[width=0.238\textwidth]{ppfigs_f023}&
\includegraphics[width=0.238\textwidth]{ppfigs_f024}
\end{tabular}
\vspace*{-0.2cm}
\caption[Expected running time divided by dimension
versus dimension]{
\label{fig:scaling}
% command defined in cocopp_commands.tex:
\bbobppfigslegend{$f_1$ and $f_{24}$}  % \algorithmA can be defined above, see above
}
% 
\end{figure*}



%%%%%%%%%%%%%%%%%%%%%%%%%%%%%%%%%%%%%%%%%%%%%%%%%%%%%%%%%%%%%%%%%%%%%%%%%%%%%%%
%%%%%%%%%%%%%%%%%%%%%%%%%%%%%%%%%%%%%%%%%%%%%%%%%%%%%%%%%%%%%%%%%%%%%%%%%%%%%%%

% Empirical Cumulative Distribution Functions (ECDFs) per function group
% for dimension 5.

%%%%%%%%%%%%%%%%%%%%%%%%%%%%%%%%%%%%%%%%%%%%%%%%%%%%%%%%%%%%%%%%%%%%%%%%%%%%%%%
\newcommand{\rot}[2][2.5]{
  \hspace*{-3.5\baselineskip}%
  \begin{rotate}{90}\hspace{#1em}#2
  \end{rotate}}
\newcommand{\includeperfprof}[1]{% include and annotate at the side
  \input{\bbobdatapath\algsfolder #1}%
  \includegraphics[height=0.24\textheight]{#1}%
  %\raisebox{.12\textheight}{
	%\parbox[b][.24\textheight]{.0868\textwidth}{\begin{scriptsize}
  %  \perfprofsidepanel % this is "\algaperfprof \vfill \algbperfprof \vfill" etc
  %\end{scriptsize}}
	%}
}
%%%%%%%%%%%%%%%%%%%%%%%%%%%%%%%%%%%%%%%%%%%%%%%%%%%%%%%%%%%%%%%%%%%%%%%%%%%%%%%
\begin{figure*}
\begin{tabular}{@{}c@{\hspace*{0.05\textwidth}}c@{}}
 separable fcts & moderate fcts \\
 \includeperfprof{pprldmany_05D_separ} &
 \includeperfprof{pprldmany_05D_lcond} \\ 
ill-conditioned fcts & multi-modal fcts \\
 \includeperfprof{pprldmany_05D_hcond} &
 \includeperfprof{pprldmany_05D_multi} \\ 
 weakly structured multi-modal fcts & all functions\\
 \includeperfprof{pprldmany_05D_mult2} & 
 \includeperfprof{pprldmany_05D_noiselessall} 
 \end{tabular}
\caption{
\label{fig:ECDFs05D}
\bbobECDFslegend{5}
}
\end{figure*}


%%%%%%%%%%%%%%%%%%%%%%%%%%%%%%%%%%%%%%%%%%%%%%%%%%%%%%%%%%%%%%%%%%%%%%%%%%%%%%%
%%%%%%%%%%%%%%%%%%%%%%%%%%%%%%%%%%%%%%%%%%%%%%%%%%%%%%%%%%%%%%%%%%%%%%%%%%%%%%%

% Empirical Cumulative Distribution Functions (ECDFs) per function group
% for dimension 20.

%%%%%%%%%%%%%%%%%%%%%%%%%%%%%%%%%%%%%%%%%%%%%%%%%%%%%%%%%%%%%%%%%%%%%%%%%%%%%%%
\begin{figure*}
 \begin{tabular}{@{}c@{\hspace*{0.05\textwidth}}c@{}}
 separable fcts & moderate fcts \\
 \includeperfprof{pprldmany_20D_separ} &
 \includeperfprof{pprldmany_20D_lcond} \\ 
ill-conditioned fcts & multi-modal fcts \\
 \includeperfprof{pprldmany_20D_hcond} &
 \includeperfprof{pprldmany_20D_multi} \\ 
 weakly structured multi-modal fcts & all functions\\
 \includeperfprof{pprldmany_20D_mult2} & 
 \includeperfprof{pprldmany_20D_noiselessall} 
 \end{tabular}
\caption{
\label{fig:ECDFs20D}
\bbobECDFslegend{20}
}
\end{figure*}


%%%%%%%%%%%%%%%%%%%%%%%%%%%%%%%%%%%%%%%%%%%%%%%%%%%%%%%%%%%%%%%%%%%%%%%%%%%%%%%
%%%%%%%%%%%%%%%%%%%%%%%%%%%%%%%%%%%%%%%%%%%%%%%%%%%%%%%%%%%%%%%%%%%%%%%%%%%%%%%

% Expected runtime (ERT in number of function evaluations)
% divided by the best ERT measured during BBOB-2009 (given in the respective
% first row) for functions $f_1$--$f_{24}$ for dimension 5.

%%%%%%%%%%%%%%%%%%%%%%%%%%%%%%%%%%%%%%%%%%%%%%%%%%%%%%%%%%%%%%%%%%%%%%%%%%%%%%%
\begin{table*}\tiny
%\hfill5-D\hfill~\\[1ex]
{\normalsize \color{red}
\ifthenelse{\isundefined{\algorithmG}}{}{more than 6 algorithms: please split the tables below by hand until it fits to the page limits}
}
\mbox{\begin{minipage}[t]{0.499\textwidth}\tiny
\centering
\pptablesheader
\input{\bbobdatapath\algsfolder pptables_f001_05D} 

\input{\bbobdatapath\algsfolder pptables_f002_05D}

\input{\bbobdatapath\algsfolder pptables_f003_05D}

\input{\bbobdatapath\algsfolder pptables_f004_05D}

\input{\bbobdatapath\algsfolder pptables_f005_05D}

\input{\bbobdatapath\algsfolder pptables_f006_05D}

\input{\bbobdatapath\algsfolder pptables_f007_05D}

\input{\bbobdatapath\algsfolder pptables_f008_05D}

\input{\bbobdatapath\algsfolder pptables_f009_05D}

\input{\bbobdatapath\algsfolder pptables_f010_05D}

\input{\bbobdatapath\algsfolder pptables_f011_05D}

\input{\bbobdatapath\algsfolder pptables_f012_05D}
\end{tabularx}

\end{minipage}
\hspace{0.002\textwidth}
\begin{minipage}[t]{0.499\textwidth}\tiny
\centering
\pptablesheader
\input{\bbobdatapath\algsfolder pptables_f013_05D}

\input{\bbobdatapath\algsfolder pptables_f014_05D}

\input{\bbobdatapath\algsfolder pptables_f015_05D}

\input{\bbobdatapath\algsfolder pptables_f016_05D}

\input{\bbobdatapath\algsfolder pptables_f017_05D}

\input{\bbobdatapath\algsfolder pptables_f018_05D}

\input{\bbobdatapath\algsfolder pptables_f019_05D}

\input{\bbobdatapath\algsfolder pptables_f020_05D}

\input{\bbobdatapath\algsfolder pptables_f021_05D}

\input{\bbobdatapath\algsfolder pptables_f022_05D}

\input{\bbobdatapath\algsfolder pptables_f023_05D}

\input{\bbobdatapath\algsfolder pptables_f024_05D}
\end{tabularx}
\end{minipage}}

 \caption{\label{tab:ERTs5}
 \bbobpptablesmanylegend{dimension $5$}
 }
\end{table*}


%%%%%%%%%%%%%%%%%%%%%%%%%%%%%%%%%%%%%%%%%%%%%%%%%%%%%%%%%%%%%%%%%%%%%%%%%%%%%%%
%%%%%%%%%%%%%%%%%%%%%%%%%%%%%%%%%%%%%%%%%%%%%%%%%%%%%%%%%%%%%%%%%%%%%%%%%%%%%%%

% Expected runtime (ERT in number of function evaluations)
% divided by the best ERT measured during BBOB-2009 (given in the respective
% first row) for functions $f_1$--$f_{24}$ for dimension 20.

%%%%%%%%%%%%%%%%%%%%%%%%%%%%%%%%%%%%%%%%%%%%%%%%%%%%%%%%%%%%%%%%%%%%%%%%%%%%%%%
\begin{table*}\tiny
%\hfill20-D\hfill~\\[1ex]
\mbox{\begin{minipage}[t]{0.499\textwidth}\tiny
\centering
\pptablesheader
\input{\bbobdatapath\algsfolder pptables_f001_20D} 

\input{\bbobdatapath\algsfolder pptables_f002_20D}

\input{\bbobdatapath\algsfolder pptables_f003_20D}

\input{\bbobdatapath\algsfolder pptables_f004_20D}

\input{\bbobdatapath\algsfolder pptables_f005_20D}

\input{\bbobdatapath\algsfolder pptables_f006_20D}

\input{\bbobdatapath\algsfolder pptables_f007_20D}

\input{\bbobdatapath\algsfolder pptables_f008_20D}

\input{\bbobdatapath\algsfolder pptables_f009_20D}

\input{\bbobdatapath\algsfolder pptables_f010_20D}

\input{\bbobdatapath\algsfolder pptables_f011_20D}

\input{\bbobdatapath\algsfolder pptables_f012_20D}
\end{tabularx}
\end{minipage}
\hspace{0.002\textwidth}
\begin{minipage}[t]{0.499\textwidth}\tiny
\centering
\pptablesheader
\input{\bbobdatapath\algsfolder pptables_f013_20D}

\input{\bbobdatapath\algsfolder pptables_f014_20D}

\input{\bbobdatapath\algsfolder pptables_f015_20D}

\input{\bbobdatapath\algsfolder pptables_f016_20D}

\input{\bbobdatapath\algsfolder pptables_f017_20D}

\input{\bbobdatapath\algsfolder pptables_f018_20D}

\input{\bbobdatapath\algsfolder pptables_f019_20D}

\input{\bbobdatapath\algsfolder pptables_f020_20D}

\input{\bbobdatapath\algsfolder pptables_f021_20D}

\input{\bbobdatapath\algsfolder pptables_f022_20D}

\input{\bbobdatapath\algsfolder pptables_f023_20D}

\input{\bbobdatapath\algsfolder pptables_f024_20D}
\end{tabularx}
\end{minipage}}
 \caption{\label{tab:ERTs20}
  \bbobpptablesmanylegend{dimension $20$}
}
\end{table*}


%%%%%%%%%%%%%%%%%%%%%%%%%%%%%%%%%%%%%%%%%%%%%%%%%%%%%%%%%%%%%%%%%%%%%%%%%%%%%%%

\bibliographystyle{ACM-Reference-Format}
\bibliography{./bbob}  % bbob.bib is the name of the Bibliography in this case

\clearpage % otherwise the last figure might be missing


\end{document}
